%======================================================%
% Perimeter Scholars International Essay Template 2019 %
% Author: Agata Branczyk, January 2019                 %
% Last updated: 31 January 2019 by Agata Branczyk      %
%======================================================%

\documentclass[12pt,twoside]{book}

%%%%%%%%%%%%%%%%%%%%%%%%%%%%%%%%%%%%%%%%%%%%%%%%%%%%%%%%%%%%%
%% EDIT THESE COMMANDS BELOW TO REFLECT YOUR ESSAY PROJECT %%
%% (this will automatically populate the entire document)  %%
%%%%%%%%%%%%%%%%%%%%%%%%%%%%%%%%%%%%%%%%%%%%%%%%%%%%%%%%%%%%%

\newcommand{\essaytitle}{Your PSI Essay Title}
\newcommand{\shortessaytitle}{Your (short) PSI Essay Title} % Make this shorter if you see it spilling over within the header
\newcommand{\yourname}{Your name}
\newcommand{\yoursupervisor}{Your Supervisor's name/s}

%%%%%%%%%%%%%%%%%%%%%%%%%%%%%%%%%%
%% TEMPLATE FILES - DO NOT EDIT %%
%%%%%%%%%%%%%%%%%%%%%%%%%%%%%%%%%%
%%%%%%%%%%%%%%%%%%%%%%%%%%%
%% DO NOT EDIT THIS FILE %%
%%%%%%%%%%%%%%%%%%%%%%%%%%%

\usepackage[utf8]{inputenc}
\usepackage[english]{babel}
\usepackage{fancyhdr}
\usepackage{lipsum}
\usepackage{graphicx}
\usepackage{titlesec}
\usepackage{appendix}
\usepackage[sectionbib]{chapterbib}
\usepackage[breakwords]{truncate}
\usepackage{lastpage}
\usepackage{amsmath}
\usepackage{hyperref}
\usepackage{libertine}
\usepackage[libertine,cmintegrals,cmbraces,vvarbb]{newtxmath}
\usepackage[font={small}]{caption}

%%%%%%%%%%%%%%%%%%%%%%%%%%%%%%%%%%%%%%
%% You may add your packages in the %%
%% main text, or in a separate file %%
%%%%%%%%%%%%%%%%%%%%%%%%%%%%%%%%%%%%%%
%%%%%%%%%%%%%%%%%%%%%%%%%%%
%% DO NOT EDIT THIS FILE %%
%%%%%%%%%%%%%%%%%%%%%%%%%%%

% Makes the sections have chapter numbering
\renewcommand*\thesection{\arabic{section}}

% Renames "Bibliography" to "References"
 \usepackage[nottoc,notlof,notlot]{tocbibind} 
 \addto\captionsenglish{\renewcommand{\bibname}{References}}

% Formatting for the title on page 1
\titleformat{\chapter}[display]
  {\bfseries\large}{}{-10ex}
  {\titlerule\vspace{2ex}\filright\Huge}
  [\vspace{1ex}\titlerule]

% Headers, footers, and page numbers 
\pagestyle{fancy}
\fancyhf{}
\renewcommand\sectionmark[1]{%
   \markright{\thesection\ ~~#1}}% gets rid of the dot in the section number in the header
\fancyhead[CE]{\S \emph{\rightmark}} % makes the section title header
\fancyhead[CO]{\emph{\expandafter\MakeUppercase\expandafter{\shortessaytitle}}} % makes the essay title header
\fancyhead[LE,RO]{\textbf{\thepage}} % makes the page numbers header

% Sets the margins
\setlength{\oddsidemargin}{1.4cm}
\setlength{\evensidemargin}{1.4cm}


\begin{document}
%%%%%%%%%%%%%%%%%%%%%%%%%%%%%%
%% TITLE PAGE - DO NOT EDIT %%
%%%%%%%%%%%%%%%%%%%%%%%%%%%%%%
% \setcounter{page}{1} % sets the page number for the book compilation (ignore this)
%%%%%%%%%%%%%%%%%%%%%%%%%%%
%% DO NOT EDIT THIS FILE %%
%%%%%%%%%%%%%%%%%%%%%%%%%%%

\setlength{\headheight}{15pt} 

% Title page
\begin{center}
\includegraphics[height=0.35\textwidth]{./style_files/PSILOGO.pdf}\\[2cm]
\textbf{\huge\essaytitle}\\[3cm]
\textbf{\Large\yourname}\\
\vspace*{\fill}
 \large{\textbf{An essay submitted}\\ 
  \textbf{for partial fulfillment of}\\ 
  \textbf{Perimeter Scholars International}\\[2cm]
  \textbf{June, 2019}}
\end{center}

\pagestyle{empty} % leaves a the next page black

\frontmatter % makes Roman page numbers
\pagestyle{fancy} % undoes the empty page style

% Table of contents
\tableofcontents
\mainmatter % makes Arabic page numbers

% % The heading on the front page
\chapter*{\essaytitle} 

\vspace{-0.5cm}
\centerline{\textbf{\yourname}} 
\vspace{0.3cm}
\centerline{Supervisor: \yoursupervisor} 
\vspace{0.5cm} 



%%%%%%%%%%%%%%%%%%%%%%%%%%%%%%%%
%% ESSAY ABSTRACT - EDIT THIS %%
%%%%%%%%%%%%%%%%%%%%%%%%%%%%%%%%
\begin{quote}
The abstract is an important component of your essay. It is likely the first substantive description of your work read by an external examiner. You should view it as an opportunity to set accurate expectations. The abstract is a summary of the whole thesis. It presents all the major elements of your work in a highly condensed form. An abstract is not merely an introduction in the sense of a preface, preamble, or advance organizer that prepares the reader for the thesis. It must also be capable of substituting for the whole thesis when there is insufficient time and space for the full text.
\end{quote}

%%%%%%%%%%%%%%%%%%%%%%%%%%%%%%%%%%%%%%%%%%%%
%% YOUR ESSAY CHAPTERS - EDIT THESE FILES %%
%%%%%%%%%%%%%%%%%%%%%%%%%%%%%%%%%%%%%%%%%%%

\section{Introduction}
 
\textbf{You can cite books \cite{Feynman1998} and papers \cite{Lee2016} like this, and they appear in the references section toward the end of the document. (Click on the reference number and it will take you there!)}

\lipsum[5-7] % generic text; delete this and replace with your text

\textbf{You can also include and reference figures, for example, see Figure \ref{fig:example}.}
 
\begin{figure}[h]
    \centering
    \includegraphics[width=0.9\textwidth]{./images/phd022410s.jpg}
    \caption{Here is a comic from \href{http://phdcomics.com/comics/archive.php?comicid=1285}{PhDComics}. (The previous word is also a clickable link! But if you click it, don't fall down the rabbit hole reading comics rather than writing your essay!). If you get stuck procrastinating, read this article as an antidote: \href{https://www.waitbutwhy.com/2013/10/why-procrastinators-procrastinate.html}{Why procrastinators procrastinate}. }
    \label{fig:example}
\end{figure}


\input{./chapters/chapter2}
\section{Results}
 
\lipsum[43] % generic text; delete this and replace with your text

\subsection{These are the things that I learnt} 
 
\lipsum[44] % generic text; delete this and replace with your text

\section{Conclusion}
 
In conclusion, here are some more references \cite{Pincus2016,Hardy2016,Einstein1906,Turok1996}. 
 
\lipsum[53-55] % generic text; delete this and replace with your text

% You can add/remove chapters

\section{Acknowledgements}

Acknowledgements aren't mandatory, but it is always nice to thank people that helped you with your project (both, directly and indirectly).

%%%%%%%%%%%%%%%%%%%%%%%%%%%%%%%%%%%%%%%
%% THE REFERENCES - EDIT THESE FILES %%
%%%%%%%%%%%%%%%%%%%%%%%%%%%%%%%%%%%%%%%
\bibliographystyle{style_files/bibstyleNCM}
\bibliography{references}

%%%%%%%%%%%%%%%%%%%%%%%%%%%%%%%%%%%%%%%
%% THE APPENDICES - EDIT THESE FILES %%
%%%%%%%%%%%%%%%%%%%%%%%%%%%%%%%%%%%%%%%
\begin{subappendices} % DO NOT EDIT THIS LINE
\renewcommand\thesection{\Alph{section}} % DO NOT EDIT THIS LINE

\section{One appendix}

\lipsum[5] % generic text; delete this and replace with your text

\subsection{And a subappendix}

\lipsum[6] % generic text; delete this and replace with your text

\section{Another appendix}

\lipsum[33] % generic text; delete this and replace with your text

% You can add/remove appendices

\end{subappendices} % DO NOT EDIT THIS LINE

% \cleardoublepage % makes sure there is an even number of pages for the book compilation (ignore this)

\end{document}